%% LaTeX Beamer presentation template (requires beamer package)
%% see http://bitbucket.org/rivanvx/beamer/wiki/Home
%% idea contributed by H. Turgut Uyar
%% template based on a template by Till Tantau
%% this template is still evolving - it might differ in future releases!

\documentclass{beamer}

\mode<presentation>
{
\usetheme{Warsaw}
% \usetheme{Copenhagen}
\usecolortheme{beaver}

\setbeamercovered{transparent}
}

\usepackage[english]{babel}
\usepackage[latin1]{inputenc}

% font definitions, try \usepackage{ae} instead of the following
% three lines if you don't like this look
\usepackage{mathptmx}
\usepackage[scaled=.90]{helvet}
\usepackage{courier}

\usepackage[T1]{fontenc}

% User packages
\usepackage[absolute,overlay]{textpos}
\usepackage{tikz}
\usepackage{listings}

%%% Global Settings %%%%%%%%%%%%%%%%%%%%%%%%%%%%%%%%%%%%%%%%%%%%%%%%%%%%%%%%%%%

\graphicspath{{images/}{draw.io/}} % Root directory of the pictures

%%% Title %%%%%%%%%%%%%%%%%%%%%%%%%%%%%%%%%%%%%%%%%%%%%%%%%%%%%%%%%%%%%%%%%%%%%

\title{CP4CDS}

\subtitle{Climate Model Data and Compute for Copernicus Climate Data Store}

% - Use the \inst{?} command only if the authors have different
%   affiliation.
\author{\vspace{2.3cm}\\
Carsten Ehbrecht\inst{1}
\and Ruth Petrie\inst{2}
\and Ag Stephens\inst{2}
}
%\author{\inst{1}}

% - Use the \inst command only if there are several affiliations.
% - Keep it simple, no one is interested in your street address.
\institute[Institute]
{
\inst{1}%
DKRZ - German Climate Compute Center
\and
\inst{2}%
STFC - Science and Technology Facilities Council
}

\date{\footnotesize{$1^{st}$ of February 2018 / MAGIC Workshop at KNMI}}


% This is only inserted into the PDF information catalog. Can be left
% out.
\subject{Talks}



% If you have a file called "university-logo-filename.xxx", where xxx
% is a graphic format that can be processed by latex or pdflatex,
% resp., then you can add a logo as follows:

% \pgfdeclareimage[height=0.5cm]{university-logo}{university-logo-filename}
% \logo{\pgfuseimage{university-logo}}



% Delete this, if you do not want the table of contents to pop up at
% the beginning of each subsection:
\AtBeginSubsection[]
{
\begin{frame}<beamer>
\frametitle{Outline}
\tableofcontents[currentsection,currentsubsection]
\end{frame}
}

% Section title slides
\AtBeginSection[]{
  \begin{frame}
  \vfill
  \centering
  \begin{beamercolorbox}[sep=8pt,center,shadow=true,rounded=true]{title}
    \usebeamerfont{title}\insertsectionhead\par%
  \end{beamercolorbox}
  \vfill
  \end{frame}
}


% If you wish to uncover everything in a step-wise fashion, uncomment
% the following command:

%\beamerdefaultoverlayspecification{<+->}

\begin{document}

\begin{frame}
   \tikz [remember picture,overlay]
    \node at
        ([yshift=4.8cm]current page.south)
        %or: (current page.center)
        {\includegraphics[height=1cm]{copernicus-logo} \includegraphics[height=1cm]{c3s-logo}};
   \titlepage
\end{frame}


\begin{frame}
\frametitle{Outline}
\tableofcontents
% You might wish to add the option [pausesections]
\end{frame}


% %%%%%%%%%%%%%%%%%%%%%%%%%%%%%%%%%%%%%%%%%%%%%%%%%%%%%%%%%%%%%%%%%%%%%%%%%%%%%
\section{Overview}

% -----------------------------------------------
\begin{frame}
\frametitle<presentation>{CP4CDS Services: Data + Compute}

  \begin{figure}[ht]
    \centering
    \includegraphics[height=5cm]{cp4cds.png}
  \end{figure}

  \centering
  \footnotesize{\url{https://github.com/cp4cds}}

\end{frame}

% -----------------------------------------------
\begin{frame}
\frametitle<presentation>{CP4CDS Service Interfaces}

  \begin{figure}[ht]
    \centering
    \includegraphics[height=5cm]{cp4cds-interfaces.png}
  \end{figure}

\end{frame}

% -----------------------------------------------
\begin{frame}
\frametitle<presentation>{CP4CDS Federated Services}

  \begin{figure}[ht]
    \centering
    \includegraphics[height=4cm]{cp4cds-federated.png}
  \end{figure}

  \begin{itemize}
    \item WPS outputs are available on a file-service but can be served by other protocols like OpenDAP, WMS etc.
    \item ESGF NetCDF files are served via a thredds server, file-service and OpenDAP are enabled by default.
      Other service protocols can be enabled by as well, like WMS, WCS etc.
  \end{itemize}

\end{frame}

% %%%%%%%%%%%%%%%%%%%%%%%%%%%%%%%%%%%%%%%%%%%%%%%%%%%%%%%%%%%%%%%%%%%%%%%%%%%%%
\section{Data Service}

% -----------------------------------------------
\begin{frame}
\frametitle<presentation>{Climate Model Data}

  \begin{itemize}
    \item CMIP5 subset selected for Copernicus.
    \item Quality-checked data.
    \item CORDEX is on its way ...
  \end{itemize}

\end{frame}

% -----------------------------------------------
\begin{frame}
\frametitle<presentation>{ESGF Data Node}

  \begin{figure}[ht]
    \centering
    \includegraphics[height=4cm]{cp4cds-data-node-at-ceda}
  \end{figure}

  \begin{itemize}
    \item Using vanilla ESGF data nodes.
    \item Federated between CEDA, IPSL and DKRZ.
  \end{itemize}

  \centering
  \footnotesize{\url{https://cp4cds-index1.ceda.ac.uk/projects/cp4cds_ceda/}}

\end{frame}

% %%%%%%%%%%%%%%%%%%%%%%%%%%%%%%%%%%%%%%%%%%%%%%%%%%%%%%%%%%%%%%%%%%%%%%%%%%%%%
\section{Compute Service}

% -----------------------------------------------
\begin{frame}
\frametitle<presentation>{CP4CDS Compute Service}

  \begin{figure}[ht]
    \centering
    \includegraphics[height=6cm]{cp4cds-wps}
  \end{figure}

  \centering
  \footnotesize{\url{https://github.com/cp4cds/copernicus-wps-demo}}

\end{frame}

% -----------------------------------------------
\begin{frame}
\frametitle<presentation>{C3S Magic: Climate Diagnostics}

  \begin{figure}[ht]
    \centering
    \includegraphics[height=5cm]{c3s-magic}
  \end{figure}

  \centering
  \footnotesize{\url{https://github.com/c3s-magic}}

\end{frame}

% -----------------------------------------------
\begin{frame}
\frametitle<presentation>{SDDS - Software Dependency and Deployment Solution}

  \begin{figure}[ht]
    \centering
    \includegraphics[height=6cm]{cp4cds-toolbox}
  \end{figure}

  \centering
  \footnotesize{\url{https://github.com/cp4cds/cp4cds-wps-template}}

\end{frame}

% -----------------------------------------------
\begin{frame}
\frametitle<presentation>{Deployment Tools}
  \begin{columns}[c]
    \begin{column}{.2\textwidth}
      \includegraphics[width=0.6\textwidth]{ansible}
    \end{column}
    \begin{column}{.8\textwidth}
      \begin{itemize}
        \item Using Ansible to deploy CP4CDS WPS service.
        \item Deployment scenarios: single host, cluster, docker orchestration.
        \item Ansible will replace the current Buildout deployment solution (application level, single host).
        \item \footnotesize{\url{https://github.com/bird-house/birdhouse-ansible/tree/master/demo}}
      \end{itemize}
    \end{column}
  \end{columns}
  % next row
  \vrule
  \begin{columns}[c]
    \begin{column}{.2\textwidth}
      \includegraphics[width=\textwidth]{conda_logo}
    \end{column}
    \begin{column}{.8\textwidth}
      \begin{itemize}
        \item All Application and Diagnostics software dependencies are managed with Conda.
        \item Production-ready packages are maintained on conda-forge community channel.
        \item \footnotesize{\url{https://conda.io/docs/index.html}}
      \end{itemize}
    \end{column}
  \end{columns}
  % next row
  \vrule
  \begin{columns}[c]
    \begin{column}{.2\textwidth}
      \includegraphics[width=\textwidth]{docker}
    \end{column}
    \begin{column}{.8\textwidth}
      \begin{itemize}
        \item Kubernetes Container Orchestration
        \item Planned: PyWPS Docker Extension to launch WPS Jobs as Container
      \end{itemize}
    \end{column}
  \end{columns}

\end{frame}


% -----------------------------------------------
\begin{frame}
\frametitle<presentation>{PyWPS}

  \begin{figure}[ht]
    \centering
    \includegraphics[height=2cm]{pywps}
  \end{figure}

  \begin{itemize}
    \item An implementation of the OGC Web Processing Service standard
    \item Implements WPS 1.0.0 standard (WPS 2.0.0 in progress)
    \item Coded in the Python language (researcher friendly)
    \item Easy to hack (developer friendly)
    \item Relevant contributions by over a dozen individuals
    \item OSGeo accreditation around the corner \ldots
  \end{itemize}

  \vspace{0.2cm}
  \centering
  \footnotesize{\url{http://pywps.org}}

\end{frame}

% -----------------------------------------------
\begin{frame}
\frametitle<presentation>{PyWPS Scheduler Extension}

  \begin{figure}[ht]
    \centering
    \includegraphics[height=4cm]{pywps-scheduler-extension}
  \end{figure}

  \centering
  \footnotesize{\url{https://github.com/geopython/pywps/blob/develop/docs/extensions.rst}}

\end{frame}

% -----------------------------------------------
\begin{frame}
\frametitle<presentation>{WPS: Providing Optional Metadata}

  \begin{itemize}
    \item Using WPS Metadata attribute to point to additinal information for input/output parameters.
    \item Role attribute indicates the semantic of the provided link.
    \item Additional information can be provided by a WPS meta-helper process.
  \end{itemize}

\end{frame}

% -----------------------------------------------
\begin{frame}
\frametitle<presentation>{WPS Clients for Demo and Testing}
\begin{columns}[c]
  \begin{column}{.5\textwidth}
    \includegraphics[width=\textwidth]{magic-portal}
  \end{column}
  \begin{column}{.5\textwidth}
    \includegraphics[width=\textwidth]{phoenix}
  \end{column}
\end{columns}
% next row
\vrule
\begin{columns}[c]
  \begin{column}{.5\textwidth}
    \includegraphics[width=\textwidth]{birdy-terminal}
  \end{column}
  \begin{column}{.5\textwidth}
    \includegraphics[width=\textwidth]{birdy-terminal}
  \end{column}
\end{columns}

\end{frame}

% %%%%%%%%%%%%%%%%%%%%%%%%%%%%%%%%%%%%%%%%%%%%%%%%%%%%%%%%%%%%%%%%%%%%%%%%%%%%%
\section{Summary}

\begin{frame}
\frametitle<presentation>{Summary}

\begin{itemize}

  \item Deployment
  \begin{itemize}
    \item Nginx + Gunicorn provide the infrastructure for scalable services
    \item Birdhouse supports automatic deployment using Conda and Ansible
  \end{itemize}

  \item{Toolbox}
  \begin{itemize}
    \item Web portal and command-line tool for testing and demo of WPS services
    \item Security middleware to protect the execution of WPS processes
  \end{itemize}

  \item{Get your hands dirty}
  \begin{itemize}
    \item Birdhouse Workshop: \url{http://birdhouse-workshop.readthedocs.io/en/latest/index.html}
    \item PyWPS Workshop: \url{https://github.com/PyWPS/pywps-workshop}
  \end{itemize}
\end{itemize}

\end{frame}

% %%%%%%%%%%%%%%%%%%%%%%%%%%%%%%%%%%%%%%%%%%%%%%%%%%%%%%%%%%%%%%%%%%%%%%%%%%%%%
\appendix

\section{Appendix}

% -----------------------------------------------
\begin{frame}
\frametitle<presentation>{The OGC Web Processing Service}

  \begin{itemize}
    \item OGC open web standard for remote geo-spatial processing.
  \end{itemize}
\end{frame}

\end{document}
